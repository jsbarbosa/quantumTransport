\documentclass[11pt]{article}

\usepackage[margin = 3cm]{geometry}
\usepackage{braket}

\begin{document}
	{
		\centering
		\scshape
		\LARGE
		Universidad de los Andes
		
		\large
		Departamento de F\'isica
		
		Mec\'anica Cu\'antica II
		
		\vspace{1cm}
		\Large
		Teleportaci\'on cu\'antica
		
		\normalsize
		Mar\'ia Fernanda G\'omez, Juan Barbosa
		
		\vspace{1cm}
	}
	
	La paradoja EPR (Einstein, Podolsky, Rosen) fue descrita por primera vez en 1935 como un experimento mental, en el que usando las propiedades de la mec\'anica cu\'antica es posible transmitir informaci\'on a una velocidad mayor a la de la luz. Si bien para la \'epoca lo anterior parec\'ia indicar que la teor\'ia estaba incompleta y que la interpretaci\'on de Copenhagen era incorrecta, hoy se considera la interpretaci\'on como correcta y la soluci\'on a la paradoja relacionada con el entrelazamiento cu\'antico, el cual requiere de cualquier forma un canal de comunicaci\'on cl\'asico, el cual est\'a limitado a la velocidad de la luz.
	
	La teleportaci\'on cu\'antica consiste el uso de la paradoja EPR para asistir en el env\'io de un estado cu\'antico de una ubicaci\'on a otra. Para esto la informaci\'on de un estado cu\'antico $\ket{\phi_0}$ puede ser dividida un dos partes, una puramente cl\'asica y otra no cl\'asica. En el proceso el estado $\ket{\phi_0}$ es destru\'ido y s\'olamente cuando se ha recibido la informaci\'on de ambos canales, es posible reconstruir el estado $\ket{\phi_0}$. Lo anterior permite que no se violen los teoremas de no clonaci\'on y la teor\'ia de la relatividad.
	
	\nocite{*}
	\bibliography{bibliography}
	\bibliographystyle{plain}
\end{document}
